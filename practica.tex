\documentclass[a4paper,12pt]{report}
\usepackage{config}

% Description
\newcommand{\authorName}{GRAMA Alina}
\newcommand{\documentTitle}{Raport la practică de producție}
\newcommand{\uniGroupName}{DJ2203}
\newcommand{\thesisType}{licență}
\newcommand{\programulDeStudii}{licență}
\newcommand{\identificatorulCursului}{0211.7 Designul jocurilor}

% Hardcoded for now
\newcommand{\anexeCount}{20}
\newcommand{\conferencesList}{I don't know what to write here}

\begin{document}

% Prevent spills over margin
\sloppy

\begin{titlepage}
  \vspace*{\fill}
  \begin{center}
      \vspace*{1cm}

      \large
      \uppercase{\textbf{UNIVERSITATEA DE STAT DIN MOLDOVA\\}}

      \normalsize
      \uppercase{\textbf{FACULTATEA DE MATEMATICĂ SI INFORMATICĂ}}
      \vspace{0.1cm}

      \normalsize
      \uppercase{\textbf{SPECIALITATEA INFORMATICA APLICATĂ}}
      \vspace{3.0cm}

      \large
      \textbf{\uppercase\expandafter{\authorName}}
      \vspace{1.5cm}

      \Large
      \textbf{\uppercase\expandafter{\documentTitle}}
      \vspace{0.75cm}

    \end{center}
  \vfill

  \normalsize

  \begin{flushleft}
    \begin{tabular}{ p{4cm} p{4cm} p{8cm}}
      Conducătorul științific: & \underscores{4cm} & CURMANSCHII Anton,\\
                               &                   & conferențiar universitar\\ % изменить
      Autor:                   & \underscores{4cm} & \authorName,\\
                               &                   & studenta din grupa DJ2203\\
    \end{tabular}
  \end{flushleft}

  \vspace{1cm}

  \begin{center}
    \textbf{Chișinău -- 2025}
  \end{center}

\end{titlepage}

\clearpage
\tableofcontents

\clearpage
\unnumberedChapter{Список аббревиатур}
\begin{acronym}
  \acro{API}{Application Programming Interface}
  \acro{DI}{Dependency Injection}
  \acro{EF}{Entity Framework}
  \acro{REST}{Representational State Transfery}
  \acro{HTTP}{ HyperText Transfer Protocol}
  \acro{HTTPS}{PHyperText Transfer Protocol Secure}
  \acro{URL}{Uniform Resource Locator}
  \acro{ORM}{Object-Relational Mapping}
  \acro{CRUD}{Create Read Update Delete}
  \acro{JSON}{JavaScript Object Notation}
  \acro{SQL}{Structured Query Language}
  \acro{DTO}{Data Transfer Object}
  \acro{}{}
\end{acronym}

\chapter{Детали места проведения практики}

\section{Лабораторный кабинет AVR}

Практика проходила в университете, в 423 кабинете.
Данный лабораторный кабинет представляет собой довольно большой и просторный зал с более 15 современными и мощными компьютерами.

Доступ в этот кабинет был почти каждый день с 11:00 до 20:00, где у меня была возможность использовать один из компьютеров.

\section{Процесс работы в лабораторном кабинете}

Мне было дано задание --- начать изучение технологий, которые будут использованы для создания приложения, разработать основу приложения, 
а также начать писать теоретическую часть для дипломной работы.
В моем распоряжении было все необходимое оборудование, в моем случае компьютер, для создания приложения для дипломной работы, 
а также, я могла задавать вопросы своему руководителю практики. Так как тема моей дипломной работы предполагает создание веб-\ac{API} на C\#, 
то я использовала только компьютер, а потому что в данном лабораторном кабинете используются современные компьютеры, 
я могла эффективно работать над моим заданием. 

График работы над заданием был свободным --- я могла прийти или не прийти в лабораторный кабинет, главное --- выполнение и сдача задания в срок.
При возможности я приходила и работала с начала открытия и примерно до закрытия лабораторного кабинета, если не было у меня возможности работать в кабинете, 
то я работала со своего персонального компьютера.

Более детальной разработкой плана создания моего проекта занималась я одна, поэтому вопросы, которые я задала руководителю практики были по 
структуре и форматированию дипломной работы.

\chapter{Практическое задание}

Моим практическим заданием было создать основу для моей дипломной работы, чтобы я смогла эффективно написать теоретическую часть и начать 
программировать приложение, а также начать непосредственно работу над дипломной работой.


\section{Представление темы дипломной работы}

Тема моей дипломной работы звучит так - Разработка веб-\ac{API} на C\#.
Данная тема предполагает изучение, понимание и создание проекта веб-\ac{API}.

Таким образом, дипломную работу можно разделить на 3 основные части:
\begin{enumerate}
  \item Изучение создания веб-\ac{API}.
  \item Планирование создания приложения.
  \item Реализация основы приложения.
\end{enumerate}

\section{Дополнительные задания}

Кроме самого написания дипломной работы, хотелось бы иметь систему, которая могла бы:
\begin{enumerate}
  \item Легко и быстро делать изменения, вносить правки в теоретическую часть дипломной работы и сразу видеть результаты.
  \item Программировать и проводить отладку кода приложения.
  \item Иметь возможность работать с любого компьютера.
\end{enumerate}

\chapter{Выполнение практического задания}

\subsection{Настройка среды разработки}

Эффективная разработка приложения предполагает, также, и использование некоторых инструментов, которые помогут в написании приложения.
В данном случае, так как был выбран язык программирования C\#, то требуется среда разработки, которая должна поддерживать данный язык программирования.

В официальной документации сказано, что можно использовать либо Visual Studio, либо Visual Studio Code. Я выбрала среду разработки VS Code. 
Для работы с C\# в VS Code необходимо установить соответствующие расширения, что я и сделала.

К тому же для написания моего приложения был выбран фреймворк ASP.NET Core, так как данный фреймворк является стандартом для написания веб-\ac{API} проектов. 

Также, моя дипломная работа предполагает использование баз данных, для этого был выбран MS SQL Server, для которого, тоже, требуется расширение в VS Code. 
Кроме этого, в проект был добавлен Entity Framework Core.

\section{Описание веб-\ac{API} на высоком уровне}

Во время практики изучалось создание веб-\ac{API}. Далее следуют его краткое описание.

\subsection{Цель веб-\ac{API}}

Цель разработки веб-\ac{API} --- создание удобного и стандартизированного способа взаимодействия клиента с сервером. 
Веб-\ac{API} позволяет получать, обрабатывать и передавать данные, а также интегрироваться с другими внешними сервисами. 
Его использование помогает разделить логику приложения, улучшает масштабируемость и упрощает поддержку системы.

\subsection{Основные характеристики}

Основными возможностями веб-\ac{API} являются:
\begin{itemize}
  \item
      \textbf{Использование протокола \ac{HTTP}} - передача данных происходит по \ac{HTTP}/\ac{HTTPS}.
  \item
      \textbf{\ac{REST}ful-архитектура} --- соблюдает принципы \ac{REST}.
  \item
      \textbf{Методы \ac{HTTP}} --- поддерживает GET, POST, PUT, DELETE.
  \item
      \textbf{Разделение клиентской и серверной логики} - \ac{API} отделено от фронтенда, что позволяет использовать его в разных клиентах.
  \item
      \textbf{Статус-коды \ac{HTTP}} --- использует стандартные коды.
  \item
      \textbf{Гибкость} --- \ac{API} можно развернуть на разных серверах.
  \item
      \textbf{Кроссплатформенность} --- веб-\ac{API} может работать с разными клиентами.
  \item
      \textbf{Документируемость} --- можно использовать Swagger для автоматической генерации документации.
\end{itemize}

Благодаря вышеперечисленным возможностям, веб-\ac{API} являются эффективным и 
универсальным инструментом для разработки масштабируемых и гибких приложений.


\subsection{Структура веб-\ac{API}}

Контроллеры –-- это главная точка входа для \ac{API}-запросов. Они принимают \ac{HTTP}-запросы, обрабатывают их и возвращают ответ с сервера на клиент.

В коде присутствуют элементы, указывающие на то, что данный класс является \ac{API}-контроллером:
\begin{enumerate}
  \item
      Атрибут [ApiController], который указывает, что это \ac{API}-контроллер.
  \item
      Атрибут [Route("api/[controller]")], который определяет маршрут.
  \item
      \ac{API}-контроллер должен быть унаследован от класса ControllerBase.
  \item
      Перед методами добавляются атрибуты, указывающие, что метод должен обрабатывать \ac{HTTP}-запросы.
  \item
      Методы \ac{API}-контроллера обычно возвращают IActionResult или ActionResult<T>, что позволяет гибко управлять \ac{HTTP}-ответами.
  \item
      Использование зависимостей через Dependency Injection --- \ac{API}-контроллер, как правило, получает зависимости через конструктор.
  \item
      Применение атрибутов [FromBody], [FromQuery], [FromRoute] в методах класса.
  \item
      Контроллер реализует \ac{CRUD}-операции.
\end{enumerate}

\subsection{Документация веб-\ac{API}}

Документация \ac{API} --- это важная часть разработки, так как она помогает другим 
быстро разобраться, как работать с \ac{API}. С помощью правильной документации можно легко понять, какие запросы поддерживаются, 
какие параметры нужно передавать и что будет возвращено в ответе. 
Для автоматической генерации документации часто используют такие инструменты, как Swagger. 
Эти инструменты позволяют описать структуру \ac{API} в формате \ac{JSON}. Это делает интеграцию с \ac{API} 
намного проще, предоставляя примеры запросов и ответов, а также информацию о методах и статус-кодах.

\subsection{Ограничения}

Веб-\ac{API}, несмотря на свою универсальность и удобство, требуют внимательного подхода к вопросам производительности, 
управления версиями \ac{API} и интеграции с другими сервисами. Эти факторы важно учитывать 
при проектировании и развертывании веб-\ac{API}, чтобы минимизировать возможные недостатки.

\section{Планирование написания веб-\ac{API} приложения}

\subsection{Планирование веб-\ac{API} для магазина}

Для реализации веб-\ac{API} магазина была выбрана архитектура \ac{REST}, которая предоставляет простое и гибкое взаимодействие с клиентом через \ac{HTTP}-запросы. 
Основная задача веб-\ac{API} --- управление товарами: добавление новых товаров, обновление информации о товарах, удаление товаров, а также, возможность их просмотра.

В качестве основной технологии для работы с базой данных был выбран Entity Framework Core, который представляет собой \ac{ORM} для упрощенной работы с базой данных 
и автоматической генерации \ac{SQL}-запросов. База данных была выбрана \ac{SQL} Server для надежности и производительности при обработке данных.

Преимущества выбранной архитектуры и инструментов:
\begin{itemize}
  \item
      \ac{REST}ful \ac{API} позволяет легко интегрировать систему с различными клиентами и сторонними сервисами.
  \item 
      Entity Framework Core ускоряет разработку благодаря работе с данными как с объектами и поддерживании всех необходимыех функций для работы с базой данных.
  \item
      \ac{SQL} Server предоставляет надежное и масштабируемое решение для хранения данных о товарах.
\end{itemize}

\subsection{Составляющие компоненты веб-\ac{API} проекта}

Веб-\ac{API} проект состоит из нескольких ключевых компонентов:
\begin{itemize}
  \item
  \textbf{Контроллер товаров} --- отвечает за обработку запросов, связанных с товарами, таких как добавление, обновление, удаление и получение информации о товаре.
  \item 
  \textbf{Модели данных} --- представление сущности товаров, которые используются для обмена данными между контроллерами и сервисами.
  \item
  \textbf{Сервис товаров} --- реализация логики обработки данных и взаимодействия с базой данных.
  \item
  \textbf{\ac{DTO}} --- объекты передачи данных, которые используются для упрощения передачи только нужных данных между компонентами системы.
  \item
  \textbf{Контекст базы данных} -- обеспечение взаимодействия с базой данных и хранение данных о товарах.
\end{itemize}

\subsection{Идея приложения}

Основной целью веб-\ac{API} приложения является создание удобного инструмента для продавцов магазина, 
который позволит управлять ассортиментом товаров. Продавцы смогут добавлять новые товары, 
изменять существующие, удалять ненужные и просматривать информацию о товарах.

Приложение должно быть гибким и масштабируемым, позволяя легко интегрироваться с другими системами и расширяться в будущем. 
Основной акцент сделан на удобство работы с товарами, с простыми и интуитивно понятными методами \ac{API} для работы с данными товаров, 
такими как название, описание и цена.

\subsection{Реализация основы приложения}

В ходе выполнения практики была создана основа для дальнейшего развития приложения, а также были созданы сущность товаров, DTO, сервисы и репозиторий товаров, 
а также контроллер товаров.

\unnumberedChapter{Выводы}

После завершения практики была создана основа для моей дипломной работы, как теоретическая, так и практическая часть. 
В ходе работы я изучила основы разработки веб-\ac{API}, что включало знакомство с основами \ac{REST}-архитектуры, методами обработки 
\ac{HTTP}-запросов и принципами работы с базой данных через Entity Framework Core.

Для реализации практической части дипломной работы я выбрала создание веб-\ac{API} для магазина, где я должна была разработать 
функционал для управления товарами: добавление, обновление, удаление и просмотр товаров. Для этого было настроено приложение 
с использованием C\# и ASP.NET Core, а также была настроена база данных \ac{SQL} Server для хранения информации о товарах.

\newpage
\markpage{usefulStuffEnd}

% Bibliography
\bibliographystyle{plain}
\bibliography{bibliography}
\addcontentsline{toc}{chapter}{\bibname}

% Appendices
\appendix

% Number with arabic numbers instead of Roman
\renewcommand{\thechapter}{\arabic{chapter}}
% Prepend Anexa to section names, center them
\titleformat{\section}[block]{\normalfont\normalsize\bfseries\filcenter}{Anexa \thesection~}{0pt}{}

% Since we've got just a single chapter in the appedices,
% but which is also the name of the Appendix chapter, it should be omitted.
% Makes little sense, but ok I guess.
\setcounter{chapter}{1}

\end{document}